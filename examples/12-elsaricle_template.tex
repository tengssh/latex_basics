\documentclass[
preprint, % preprint, nopreprintline, review
10pt,
oneside,
onecolumn, 1p,
%twocolumn, 3p, %5p
authoryear, % authoryear, number, sort&compress
%longtitle,
times,
]{elsarticle}
\journal{Elsevier Journal}
\usepackage{hyperref}
\usepackage{url}
\usepackage{graphicx}
\usepackage{longtable}
\newtheorem{thm}{Theorem}
\newdefinition{rmk}{Remark}
\newproof{pf}{Proof}

\begin{document}
\begin{frontmatter}
\title{Title\tnoteref{t1,t2}}
\tnotetext[t1]{title footnote 1}
\tnotetext[t2]{title footnote 2}

\author[1]{Name1\corref{cor1}\fnref{fn1}}
\author[2]{Name2\fnref{fn2}}
\author[1]{Name3}
\ead{author3@email}
\ead[url]{www.author3-website.com}

\cortext[cor1]{Corresponding author}
\fntext[fn1]{1st author footnote}
\fntext[fn2]{2nd author footnote}

\affiliation[1]{organization={Affiliation1},
								 addressline={Street1},
								 postcode={101 ONE},
								 citey={City1},
								 country={Country1}}
\affiliation[2]{organization={Affiliation2},
								 addressline={Street2},
								 postcode={102 TWO},
								 citey={City2},
								 country={Country2}}
\date{\today}

\begin{abstract}
This is a structured abstract.
\begin{description}
\item[Background] Why?
\item[Purpose] Why did you do it?
\item[Method] How did you do it?
\item[Results] What did you do and what did you find out?
\item[Conclusions] What does that mean?
\end{description}
\end{abstract}

\begin{graphicalabstract}
	\includegraphics{LaTeX.png}
\end{graphicalabstract}

\begin{highlights}
\item Highlight1
\item Highlight2
\item Highlight3
\end{highlights}

\begin{keyword}
Keyword1 \sep Keyword2 \sep Keyword3
\end{keyword}

\end{frontmatter}

\section{Introduction}
\begin{quotation}
This is a quotation.
\end{quotation}
This is introduction. Here is a citation \cite{PhysRevB.78.104104}.

\section{Methods}
\subsection{Method1}
This is for method1. An inline equation $1+1\geq2$.
A numbered equation:
\begin{equation}
F=U-TS
\end{equation}

An unnumbered equation:
\[H=T+V\]

Multiple equations:
\begin{eqnarray}
\Phi=\frac{1}{c}e^{-ikx}\\
|\Phi|^2=1
\end{eqnarray}

This is a theorem:
\begin{thm}
\[(1-x)^n=1-nx+\frac{1}{2}(n-1)nx^2-\frac{1}{6}((n-2)(n-1)n)x^3+...\]
\end{thm}

\subsection{Method2}
This is for method2. This is citation in \citet{PhysRevLett.99.077601} and \citep[e.g.][p. 1]{PhysRevLett.99.077601}. 

\begin{rmk}
This is a remark.
\end{rmk}

\begin{pf}
This is a proof.
\end{pf}

\section{Results \& Discussion}
\subsection{Result1}
This is result1.
\begin{enumerate}[(Res. 1.)]
\item This is result1.1
\item This is result1.2
\end{enumerate}

\subsection{Result2}
This is result2.
Here is a long table: (singlecolumn only)
\begin{longtable}{c | c | c | c | c | c}
\caption{This is a table.}
\label{tab:table1}\\
\hline
\textbf{Item Index} & \textbf{Tag Name} & \textbf{Value1} & \textbf{Value2} & \textbf{Value3} & \textbf{Descriptions} \\
\hline
A & a & u & x & 1 & note1 \\
\hline
B & b & v & y & 2 & note2 \\
\hline
C & c & w & z & 3 & note3 \\
\hline
\end{longtable}

\subsection{Discussion}
This is discussion.

\section{Conclusion}
\paragraph{Conclusion1}
This is conclusion1.

\paragraph{Conclusion2}
This is conclusion2.

%\begin{acknowledgments}
Author1 thanks to Author2 and Author3.
%\end{acknowledgments}

\bibliographystyle{apsrev} % apsrev, apsrmp, abbrv, alpha, acm, apalike, ieeetr, plain, unsrt
\bibliography{citation1.bib}

%\onecolumngrid
\newpage

\appendix
\section*{Appendix1}
This is appendix1.

\section*{Appendix2}
This is appendix2.

\end{document}
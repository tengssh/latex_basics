\documentclass{article}

\usepackage{amsmath}

\begin{document}

Using \$ for inline formula $ f(x)=x^2 $ or \( f(x) = \frac{1}{1-x} \)

Using equation for single equation
\begin{equation}
  f(x) = x^2 \text{ with equation numbering}
\end{equation}

(asterisk for turning-off auto numbering)
\begin{equation*}
  f(x) = x^2 \mbox{ without equation numbering}
\end{equation*}
\begin{equation}
  f(x) = x^2 \mbox{ alternative: without equation numbering}\nonumber
\end{equation}

Using align for automatic alignment
\begin{align*}
  x &= 1\\
  f(x) &= 1^2 = 1
\end{align*}

Using integrals, fractions
\begin{align*}
  f(x) &= \frac{1}{1-x}\\
  g(x) &= \int^a_b x^2 dx
\end{align*}

Using matrix with scaled brackets
\begin{equation*}
\left[
\begin{matrix}
1 & 0\\
0 & 1
\end{matrix}
\right]
\end{equation*}

Using partitioned statements
\[
\sum_{n=1}^{\infty}1+x+x^2+\cdots+x^n =
  \begin{cases}
    \frac{1}{1-x}, & \text{if \(x<1\)}\\
    \text{divergent}, & \text{if \(x\geq1\)}
  \end{cases}
\]

\end{document}
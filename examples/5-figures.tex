\documentclass{article}
\usepackage{graphicx}
\usepackage{subcaption}
\usepackage{blindtext}
\usepackage{wrapfig}
\usepackage{overpic}
\usepackage{xcolor}

\begin{document}
\begin{figure}[h!]
  \includegraphics[width=\linewidth]{LaTeX.png}
  \caption{LaTeX figure}
  \label{fig:LaTeX1}
\end{figure}
Figure \ref{fig:LaTeX1} is a LaTeX figure.

\begin{figure}[h!]
  \centering
  \begin{subfigure}[b]{0.2\linewidth}
    \includegraphics[width=\linewidth]{LaTeX.png}
    \caption{1}
    \label{fig:first}
  \end{subfigure}
  \begin{subfigure}[b]{0.2\linewidth}
    \includegraphics[width=\linewidth]{LaTeX.png}
    \caption{2}
    \label{fig:second}
  \end{subfigure}
  \begin{subfigure}[b]{0.2\linewidth}
    \includegraphics[width=\linewidth]{LaTeX.png}
    \caption{3}
    \label{fig:third}
  \end{subfigure}
  \caption{Three subfigures}
  \label{fig:LaTeX2}
\end{figure}
Figure \ref{fig:third} is the subfigure \subref{fig:third} of Figure \ref{fig:LaTeX2}.

\begin{figure}[h!]
  \centering
  \hfill
  \begin{subfigure}[b]{0.2\linewidth}
    \includegraphics[width=\linewidth]{LaTeX.png}
    \caption{1}
    \label{fig:first}
  \end{subfigure}
  \hspace\fill
  \begin{subfigure}[b]{0.2\linewidth}
    \includegraphics[width=\linewidth]{LaTeX.png}
    \caption{2}
    \label{fig:second}
  \end{subfigure}
  \hspace{-1cm}
  \begin{subfigure}[b]{0.2\linewidth}
    \includegraphics[width=\linewidth]{LaTeX.png}
    \caption{3}
    \label{fig:third}
  \end{subfigure}
  \caption{Three subfigures}
  \label{fig:LaTeX3}
\end{figure}
Use hfill to fill spaces in between.

\begin{figure}[h!]
  \centering
  \begin{subfigure}[][3cm][b]{0.2\linewidth}
    \includegraphics[width=\linewidth]{LaTeX.png}
    \caption{1}
    \label{fig:first}
  \end{subfigure}
  \begin{subfigure}[][3cm][c]{0.2\linewidth}
    \includegraphics[width=\linewidth]{LaTeX.png}
    \caption{2}
    \label{fig:second}
  \end{subfigure}
  \hspace{-1cm}
  \begin{subfigure}[][3cm][t]{0.2\linewidth}
    \includegraphics[width=\linewidth]{LaTeX.png}
    \caption{3}
    \label{fig:third}
  \end{subfigure}
  \caption{Three subfigures}
  \label{fig:LaTeX4}
\end{figure}
Increase the heights of subfigures, then the vertical positions can also be changed.

\begin{figure}[hb!]
  \centering
  \begin{subfigure}[b]{\linewidth}
    \includegraphics[width=\linewidth]{LaTeX.png}
    \caption{1}
  \end{subfigure}
  \begin{subfigure}[b]{\linewidth}
    \includegraphics[width=\linewidth]{LaTeX.png}
    \caption{2}
  \end{subfigure}
  \begin{subfigure}[b]{\linewidth}
    \includegraphics[width=\linewidth]{LaTeX.png}
    \caption{3}
  \end{subfigure}
  \caption{Subfigures with page break}
  \label{fig:LaTeX5}
\end{figure}
\begin{figure}[h!]
  \ContinuedFloat
  \centering
  \begin{subfigure}[b]{\linewidth}
    \includegraphics[width=\linewidth]{LaTeX.png}
    \caption{4}
  \end{subfigure}
  \begin{subfigure}[b]{\linewidth}
    \includegraphics[width=\linewidth]{LaTeX.png}
    \caption{5}
  \end{subfigure}
  \begin{subfigure}[b]{\linewidth}
    \includegraphics[width=\linewidth]{LaTeX.png}
    \caption{6}
  \end{subfigure}
  \caption{Subfigures with page break}
  \label{fig:LaTeX5}
\end{figure}
Use ContinuedFloat to have a page break in the middle of the subfigures.

\newpage

\blindtext

\begin{wrapfigure}{i}{0pt}
    \includegraphics[width=0.25\textwidth]{LaTeX.png}
    \caption{text-flow-around figure}
\end{wrapfigure}

\blindtext

\begin{figure}[h!]
  \centering
  \begin{overpic}[grid,width=.3\linewidth]{LaTeX.png}
  		\put(0,42){\large \textbf{(a)}}
	\end{overpic}
	\begin{overpic}[width=.3\linewidth]{LaTeX.png}
  		\put(45,42){\large \textbf{(b)}}
	\end{overpic}
	\begin{overpic}[width=.3\linewidth]{LaTeX.png}
  		\put(90,42){\large \textbf{(c)}}
	\end{overpic}
  \caption{overpic example}
  \label{fig:overpic}
\end{figure}

\begin{figure}[h!]
	\centering
	\resizebox{\linewidth}{!}{
	\begin{Overpic}[abs,grid]{\begin{tabular}{p{.9\textwidth}}\vspace{1.5cm}\\\end{tabular}}
		\put(0,0){\includegraphics[height=1.5cm]{LaTeX.png}}
		\put(0,42){(a)}
		\put(110,0){\includegraphics[height=1.5cm]{LaTeX.png}}
		\put(165,42){(b)}
		\put(220,0){\includegraphics[height=1.5cm]{LaTeX.png}}
		\put(300,0){\color{blue}{\rule{0.5cm}{0.25cm}}}
		\put(310,42){(c)}
	\end{Overpic}
	}
  \caption{Overpic example}
  \label{fig:Overpic}
\end{figure}

\end{document}
\documentclass{article}

\usepackage{varioref}
\usepackage{hyperref}
\hypersetup{colorlinks=true, linkcolor=red, citecolor=green, filecolor=cyan, urlcolor=blue}
\usepackage{cleveref}
\usepackage{makeidx}
\makeindex

\begin{document}
This is a bibtex\index{bibtex} citation \cite{PhysRevB.78.104104}.
To include the reference without citing use nocite. \nocite{PhysRevLett.99.077601}
Sometimes, you need to modify the title field manually \cite{PhysRevB.82.134106_original}, because it may not be displayed correctly \cite{PhysRevB.82.134106_modified}. 

This is a footnote\index{footnote}\footnote{\label{footnote1}supplementary information}
\newpage
To refer to a footnote, using this \ref{footnote1}, and using this \vref{footnote1} when the thing referred to is on a different page. The first letter can be modified using \cref{footnote1} and \Cref{footnote1}. \\
This is the url\index{url} to \url{https://journals.aps.org/prb/}. \\
The comparison between these packages can be seen on this \href{https://tex.stackexchange.com/questions/83037/difference-between-ref-varioref-and-cleveref-decision-for-a-thesis}{page}.

\bibliographystyle{unsrt} % abbrv, alpha, acm, apalike, ieeetr, plain, unsrt
\bibliography{citation1}
%\bibliography{citation1.bib} 
%\bibliography{citation2.bibtex} 
\printindex
\end{document}
\documentclass{beamer}

\usetheme{default}
%\usetheme{Boadilla}
%\usetheme{Madrid}
%\usetheme{Montpellier}
%\usetheme{Warsaw}
%\usetheme{Copenhagen}
%\usetheme{Goettingen}
%\usetheme{Hannover}
%\usetheme{Berkeley}

\title{Beamer example}
\subtitle{A simple template}
\author{Author}
\date{\today}
\titlegraphic{Title Graph}
\logo{LaTeX Beamer Logo}

\begin{document}

\begin{frame}
\titlepage
\end{frame}

\begin{frame}{Outline}
	\tableofcontents
\end{frame}

\section{Section 1}
\begin{frame}[label={slide1}]{Slide 1}{Subtitle A}
	\begin{itemize}
	\item Item 1
	\item Item 2
	\item Item 3
	\end{itemize}
	\hyperlink{summary}{\beamerbutton{Summary $\rightarrow$}}
\end{frame}

\frame{\frametitle{Slide 2}\framesubtitle{Subtitle B}
	\begin{enumerate}
	\item Item a
	\item Item b
	\item Item c
	\end{enumerate}
}

\section{Section 2}
\frame{\frametitle{Slide 3}\framesubtitle{Subtitle C}
	\begin{description}
	\item[DFT] Density Functional Theory
	\item[MD] Molecular Dynamics
	\item[ML] Machine Learning
	\end{description}
}

\section{Section 3}
\begin{frame}[label={summary}]{Summary}
\begin{columns}
	\begin{column}{0.5\textwidth}
		\begin{table}
			\begin{tabular}{| c | c |}
			    \hline
			    A & B \\
			    \hline
			    1 & i \\
			    2 & ii \\ 
			    \hline
			\end{tabular}
			\caption{Left column}
		\end{table}
	\end{column}
	\begin{column}{0.5\textwidth}
		\begin{block}{Right column}
			This is a block.
		\end{block}
		\begin{theorem} % corollary, proof
			$x^2+y^2=z^2$
		\end{theorem}
		\hyperlink{slide1}{\beamerbutton{Slide 1 $\rightarrow$}}
	\end{column}
\end{columns}
\end{frame}

\end{document}

